% ABSTRACT

Bad network performance could hurt user experience and downgrade application reputation. The discontinuous burst of short data transmission is the most common traffic pattern in the cellular network. That implies that the devices are frequently switching between idle state and data transmission state. Based on our observations, significant latency appears during the initial period of the data transmission. To identify the root cause of the latency problem, we use diagnostic cross layer monitoring tool, \textit{Qualcomm eXtensible Diagnostic Monitor} (QxDM), to enable the visibility of detailed data link layer (i.e. \textit{Radio Link Control} (RLC) layer) data transmission information. We also designed a novel cross-layer mapping mechanism to correlate both transport layer behavior with data link layer, and found that RLC layer protocol's inactive response to packet loss leads to unnecessary delays in the both layers. We propose a RLC \textit{Fast Re-Tx (Retransmission)} mechanism to avoid sluggish reaction to packet loss, and further reduce the latency during the initial network connection. Based on our real trace analysis, the \textit{Fast Re-Tx} mechanism could reduce the latency by up to \textit{35.69\%} over initial network connection.

